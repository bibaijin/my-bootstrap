%\documentclass[14pt,xcolor=dvipsnames]{beamer}
\documentclass[xcolor=dvipsnames]{beamer}

%主题
\usetheme{Madrid}
%\usetheme{PaloAlto}
%\usetheme{Berkeley}
%\usetheme{Berlin}
%\usetheme{Boadilla}
%\usetheme{AnnArbor}
%\usetheme{CambridgeUS}
%\usetheme{Darmstadt}
%\useoutertheme{umbcfootline}
%\setbeamertemplate{items}[default]
%\setbeamertemplate{blocks}[rounded][shadow=true]
%\usecolortheme{default}
% \usecolortheme{dolphin}
\usefonttheme[onlymath]{serif}
\setbeamertemplate{enumerate items}[default]
\setbeamertemplate{itemize items}[circle]

%xeCJK相关宏包
%字体设置
\usepackage{xltxtra,fontspec,xunicode}
\usepackage[slantfont,boldfont,CJKchecksingle]{xeCJK}
\CJKsetecglue{\hskip 0.15em plus 0.05 em minus 0.05em}
\XeTeXlinebreaklocale "zh"
\XeTeXlinebreakskip=0pt plus 1pt minus 0.1pt
%中文字体
%\setCJKmainfont[BoldFont=Adobe Heiti Std]{Adobe Song Std}
\setCJKmainfont{WenQuanYi Micro Hei}
%\setCJKsansfont[BoldFont=Adobe Heiti Std]{Adobe Kaiti Std}
\setCJKsansfont{WenQuanYi Micro Hei}
%\setCJKmonofont{WenQuanYi Micro Hei Mono}
\setCJKmonofont{WenQuanYi Micro Hei Mono}
%\usefonttheme{professionalfonts}
%\usefonttheme{serif}
%定义新字体
\setCJKfamilyfont{FZLiuKai}{FZSuXinShiLiuKaiS-R-GB}
\newcommand{\FZLiuKai}{\CJKfamily{FZLiuKai}}
\setCJKfamilyfont{song}{Adobe Song Std}
\setCJKfamilyfont{kai}{Adobe Kaiti Std}
\setCJKfamilyfont{hei}{Adobe Heiti Std}
\setCJKfamilyfont{fangsong}{Adobe Fangsong Std}
\setCJKfamilyfont{lisu}{LiSu}
\setCJKfamilyfont{youyuan}{YouYuan}
\newcommand{\song}{\CJKfamily{song}}
\newcommand{\kai}{\CJKfamily{kai}}
\newcommand{\hei}{\CJKfamily{hei}}
\newcommand{\fangsong}{\CJKfamily{fangsong}}
\newcommand{\lisu}{\CJKfamily{lisu}}
\newcommand{\youyuan}{\CJKfamily{youyuan}}

\defaultfontfeatures{Mapping=tex-text}

\renewcommand\figurename{图}
\renewcommand\tablename{表}

\usepackage{setspace}
\usepackage{colortbl}
\usepackage{hyperref}
\hypersetup
%{xetex,bookmarksnumbered=true,bookmarksopen,pdfborder=1,breaklinks,colorlinks=true,linkcolor=cyan,filecolor=black,urlcolor=cyan,citecolor=green}
%{xetex,bookmarksnumbered=true,bookmarksopen,pdfborder=1,breaklinks,colorlinks=true,filecolor=black,urlcolor=cyan,citecolor=green}
{xetex,bookmarksnumbered=true,bookmarksopen,pdfborder=1,breaklinks,colorlinks=true,CJKbookmarks=true}
%\setlength{\parindent}{2em}

%插入图片
\usepackage{graphicx}
\graphicspath{{figures/}}

%可能用到的包
\usepackage{amsmath,amssymb}
\usepackage{multimedia}
\usepackage{tabularx,multirow,multicol,keystroke,subfigure,longtable}
\usepackage{diagbox}
\usepackage[all]{xy}
\usepackage[backend=biber,style=caspervector,utf8,sorting=centy]{biblatex}
\addbibresource{ref.bib}

\usepackage[absolute,overlay]{textpos}

\usepackage{tikz}
\usetikzlibrary{shapes,arrows,dsp,chains,positioning,calc,intersections,
  arrows.meta,shadings,shapes.geometric,trees}
\usepackage{circuitikz}
\usepackage{marvosym}

\usepackage{smartdiagram}

\usepackage{booktabs}

% \usepackage{enumitem}
% \setlist{leftmarginii=0em}
\setlength{\leftmarginii}{1.2em}

\usepackage{etoolbox}
\usepackage{spot}

%\usepackage{cite}
%\usepackage[numbers,sort&compress]{natbib}
%\usepackage{hypernat}
%\bibliographystyle{plain}

%在表格、图片等的标题中显示编号
\setbeamertemplate{caption}[numbered]

%插入目录
\AtBeginSection[]{
  \begin{frame}
  \frametitle{目录}
  \makebox[\linewidth][c]{
    \begin{minipage}[c]{0.85\textwidth}
      \tableofcontents[sectionstyle=show/shaded,subsectionstyle=show/show/hide]
    \end{minipage}}
  \end{frame}
}

\AtBeginSubsection[]{
  \begin{frame}
  \frametitle{目录}
  \makebox[\linewidth][c]{
    \begin{minipage}[c]{0.85\textwidth}
      \tableofcontents[sectionstyle=show/shaded,
        subsectionstyle=show/shaded/hide]
    \end{minipage}}
  \end{frame}
}

%% 清华主题
\usecolortheme[named=violet]{structure}

\setbeamercolor{title}{bg=Orchid,fg=white}

\setbeamercolor{item projected}{fg=Orchid}

\colorlet{titleleft}{Orchid}
\colorlet{titleright}{White}

\makeatletter

\patchcmd{\beamer@sectionintoc}{\vskip1.5em}{\vskip1.0em}{}{}

\pgfdeclarehorizontalshading[titleleft,titleright]{beamer@frametitleshade}
  {\paperheight}{%
  color(0pt)=(titleleft);
  color(\paperwidth)=(titleright)}

\defbeamertemplate*{frametitle}{horizontal shading}{%
  \nointerlineskip%
  \hbox{\leavevmode
    \advance\beamer@leftmargin by -12bp%
    \advance\beamer@rightmargin by -12bp%
    \beamer@tempdim=\textwidth%
    \advance\beamer@tempdim by \beamer@leftmargin%
    \advance\beamer@tempdim by \beamer@rightmargin%
    \hskip-\Gm@lmargin\hbox{%
      \setbox\beamer@tempbox=\hbox{
        \begin{minipage}[b]{\paperwidth}%
          \vbox{}\vskip .10ex%
          \leftskip0.3cm%
          \rightskip0.3cm plus1fil\leavevmode%
          \insertframetitle%
          \vskip-2.1ex%
          \ifx\insertframesubtitle\@empty%
            \strut\par%
          \else
            \par{\usebeamerfont*{framesubtitle}{\usebeamercolor[fg]
              {framesubtitle}\insertframesubtitle}\strut\par}%
          \fi%
          \nointerlineskip%
          \vbox{}%
        \end{minipage}}%
      \beamer@tempdim=\ht\beamer@tempbox%
      \advance\beamer@tempdim by 2pt%
      \begin{pgfpicture}{0pt}{0pt}{\paperwidth}{\beamer@tempdim}
        \usebeamercolor{frametitle right}
        \pgfpathrectangle{\pgfpointorigin}{\pgfpoint{\paperwidth}
          {\beamer@tempdim}}
        \pgfusepath{clip}
        \pgftext[left,base]{\pgfuseshading{beamer@frametitleshade}}
      \end{pgfpicture}
      \hskip-\paperwidth%
      \box\beamer@tempbox}%
    \hskip-\Gm@rmargin}%
  \vskip-2pt}

\defbeamertemplate*{section in toc}{my theme}{
  \leavevmode\leftskip=0.0em\large{
    \usebeamercolor[fg]{item projected}\inserttocsectionnumber.
  }
  {\usebeamercolor[fg]{normal text}\inserttocsection}\par
}

\defbeamertemplate*{subsection in toc}{my theme}{
  \vskip0.3em
  \leavevmode\leftskip=1.85em\normalsize{
    \usebeamercolor[fg]{item projected}
    {\footnotesize\raise0.9pt\hbox{$\bullet$}}
  }
  \textcolor{black}{\inserttocsubsection}\par
}

\makeatother

%%%%%%%%%%%%%%%%%%%%%%%%%%
% 自定义命令
%%%%%%%%%%%%%%%%%%%%%%%%%%
\newcommand\ytl[2]{
  \parbox[b]{6em}{\hfill{\color{cyan}\bfseries\sffamily #1}~$\cdots$~}
  \makebox[0pt][c]{$\bullet$}\vrule\quad
  \parbox[c]{8.3cm}{\vspace{7pt}\color{red!40!black!80}\raggedright\sffamily
  #2.\\[7pt]}\\[-3pt]}

\newcommand\ytlmy[2]{
  \parbox[b]{8em}{\hfill{\color{cyan}\bfseries\sffamily #1}~$\cdots$~}
  \makebox[0pt][c]{$\bullet$}\vrule\quad
  \parbox[c]{8.0cm}{\vspace{7pt}\color{red!40!black!80}\raggedright\sffamily
  #2.\\[7pt]}\\[-3pt]
}
\newcommand{\len}{\mathrm{len}}
\newcommand*\mean[1]{\overline{#1}}
\newcommand{\Var}{\operatorname{Var}}
\newcommand{\ud}{\mathrm{d}}
